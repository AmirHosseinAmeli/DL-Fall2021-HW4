% !TeX program = xelatex

\documentclass[10pt,a4paper]{article}
\usepackage[inline]{enumitem}
\usepackage{pgfplots}
\usetikzlibrary{calc}
\newcommand{\drawaline}[4]{
	\draw [extended line=1cm,stealth-stealth] (#1,#2)--(#3,#4);
}
\usepackage{float}
\usepackage{tikz}
\usetikzlibrary {positioning}
%\usepackage {xcolor}
\definecolor {processblue}{cmyk}{0.96,0,0,0}

\usetikzlibrary{automata,arrows.meta}

\usepackage{commons/course}
\usepackage{hyperref}
\usepackage{multirow}
\usepackage{graphicx}
\usepackage{neuralnetwork}

\hypersetup{
	colorlinks=true,
	linkcolor=cyan,
	filecolor=blue,      
	urlcolor=magenta,
}


%\hidesolutions

\شروع{نوشتار}

\سربرگ{تمرین سری اول}{مفاهیم اولیه}{موعد تحویل: 22 آبان}{مدرس: دکتر بیگی}

\begin{itemize}
	\small
	\setlength\itemsep{0.05em}
	\item
	مهلت ارسال پاسخ تا ساعت 23:59 روز مشخص‌شده است.
	%	\item
	%	در طول ترم امکان ارسال با تاخیر  پاسخ همه‌ی تمارین (به استثنای هفته‌ی امتحان میانترم) تا سقف پنج روز و در مجموع ۱۵ روز، وجود دارد. پس از گذشت این مدت، پاسخ‌های ارسال‌شده پذیرفته نخواهند‌بود. 
	\item
	هم‌کاری و هم‌فکری شما در انجام تمرین مانعی ندارد اما پاسخ ارسالی هر کس حتما باید توسط خود او نوشته شده‌ باشد. 
	\item
	در صورت هم‌فکری و یا استفاده از هر منابع خارج درسی، نام هم‌فکران و آدرس منابع مورد استفاده ‌برای حل سوال مورد نظر را ذکر‌ کنید.
	\item
	لطفا تصویری واضح از پاسخ سوالات نظری بارگذاری کنید. در غیر این صورت پاسخ شما تصحیح نخواهد شد.
\end{itemize}


\section*{سوالات نظری (95 نمره)}

\مسئله{(10 نمره)‌}
فرض کنید دو تاس مستقل از هم و fair و متمایز داریم. در هر بار بازی شما به عنوان بازیکن می توانید یک حدسی برای هرکدام از تاس‌ها انجام دهید و مقداری پول در جعبه قرار دهید. در صورتی که هیچ‌کدام از حدس های شما درست نباشد شما پول خود را از دست خواهید داد. در غیر این صورت شما n (=تعداد حدس‌های درست) برابر پول جعبه به همراه اصل پول خود را برنده خواهید شد.\\
برای بازی بالا،‌ برای هردست بازی،‌ امیدریاضی سود حاصله را بدست آوردید.


\حل{
	\input{answers/1-hint.tex}
}



\مسئله{(10 نمره)}
X و Y
متغیرهای تصادفی نمایی با پارامترهای
${\lambda _X} = 2,{\lambda _Y} = 5$
و مستقل از هم هستند. متغیرهای تصادفی Z و W را به این صورت تعریف میکنیم:

$$
\begin{array}{l}
	W = \max (X,Y)\\
	Z = \min (X,Y)
\end{array}
$$

تابع چگالی احتمال را برای
$S = Z + W$
بدست آورید.

\حل{

	\input{answers/2-hint.tex}
}



\مسئله{(20 نمره)}
برای متغیرهای تصادفی دلخواه X و Y عبارات زیر را اثبات نمایید:

\begin{enumerate}[label=(\alph*)]
	\item
	با فرض مستقل بودن X و Y :  
	$
	\mathbb{E}\left[ {X|Y = y} \right] = \mathbb{E}\left[ X \right]
	$
	
	\item
	$
	\mathbb{E}\left[ {\mathbb{E}\left[ {X|Y} \right]} \right] = \mathbb{E}\left[ X \right]
	$
	
	\item
	$
	{\mathop{\rm var}} (X) = \mathbb{E}\left[ {{\mathop{\rm var}} (X|Y)} \right] + {\mathop{\rm var}} \left[ {\mathbb{E}(X|Y)} \right]
	$
	
	\item
	$
	\mathbb{E}\left[ {XY} \right] = \mathbb{E}\left[ {Y\mathbb{E}\left[ {X|Y} \right]} \right]
	$
\end{enumerate}

\حل{
	\input{answers/3-hint.tex}
}

\مسئله{(20 نمره)}
 با فرض اینکه A یک ماتریس n*n معکوس پذیر می باشد،‌ موارد زیر را ثابت کنید.

\begin{enumerate}[label=(\alph*)]
	\item
	اگر $\lambda $ یکی از مقدار ویژه های ماتریس A باشد،‌ $\frac{1}{\lambda }$ نیز یکی از مقدار ویژه های ماتریس ${A^{ - 1}}$ با همان بردار ویژه خواهد بود.
	\item
	اگر A یک ماتریس مثلثی باشد،‌عناصر روی قطر ماتریس مقادیر ویژه آن خواهند بود.
	\item
	ضرب مقدار ویژه های ماتریس A برابر دترمینان ماتریس A خواهد بود.
	\item
	جمع مقدار ویژه های ماتریس A برابر $trace(A)$ خواهد بود.
\end{enumerate}

\حل{
\input{answers/4-hint.tex}
}

\مسئله{(15 نمره)}
با فرض اینکه a و x بردارهای ستونی و A ماتریس مربعی باشد،‌روابط زیر را برای مشتقات ماتریسی اثبات کنید.

\begin{enumerate}[label=(\alph*)]
	
	\item
	$
	\frac{{d{a^T}x}}{{dx}} = {a^T}
	$
	
	\item
	$
	\frac{{d{x^T}Ax}}{{dx}} = {x^T}(A + {A^T})
	$
	
	\item
	$
	\frac{{d{{\left( {{x^T}a} \right)}^2}}}{{dx}} = 2{x^T}a{a^T}
	$
	
	
\end{enumerate}

\حل{
\input{answers/5-hint.tex}
}

\مسئله{(20 نمره)}
در این سوال، هدف اصلی محاسبه MLE و MAP برای میانگین یک توزیع نرمال تک متغیره است. در این سوال ما فرض می کنیم که $N$ نمونه $x_1,\cdots,x_N$ به طور مستقل از یک توزیع نرمال با واریانس معلوم $\sigma^2$ و میانگین نامعلوم $\mu$ گرفته شده اند.

\begin{enumerate}
	\item تخمین گر MLE برای $\mu$ را محاسبه کنید. تمامی محاسبات مورد استفاده را به طور دقیق ذکر کنید.
	
	\item حال تخمین گر MAP را برای $\mu$ بدست آورید. برای این بخش فرض کنید که توزیع پیشین (Prior) برای $\mu$ خود یک توزیع نرمال با میانگین $\nu$ و واریانس $\beta^2$ باشد. تمامی محاسبات خود را به طور دقیق ذکر کنید. 
	
	راهنمایی: می‌توانید از این نکته استفاده کنید که:
	
	$$
	\beta^{2}\left(\sum_{i=1}^{N}\left(x_{i}-\mu\right)^{2}\right)+\sigma^{2}(\mu-\nu)^{2}$$
	$$
	=\left[\mu \sqrt{N \beta^{2}+\sigma^{2}}-\frac{\sigma^{2} \nu+\beta^{2} \sum_{i=1}^{N} x_{i}}{\sqrt{N \beta^{2}+\sigma^{2}}}\right]^{2}-\frac{\left[\sigma^{2} \nu+\beta^{2} \sum_{i=1}^{N} x_{i}\right]^{2}}{N \beta^{2}+\sigma^{2}}+\beta^{2}\left(\sum_{i=1}^{N} x_{i}^{2}\right)+\sigma^{2} \nu^{2}
	$$	
	
	البته توجه کنید که بدون استفاده از این راهنمایی هم امکان حل سوال وجود دارد.
\end{enumerate}

\حل{
\input{answers/6-hint.tex}
}


\section*{سوالات عملی (15 نمره)}
\مسئله{(15 نمره)}
هدف این سوال طراحی یک شبکه ساده GAN می‌باشد که بتواند تصاویر دادگان MNIST را تولید نماید. فایل GAN.ipynb را براساس موارد خواسته شده تکمیل نمایید. دقت کنید که بخش هایی از نمره بستگی به نتایج بدست آمده دارند. باتوجه به نکات تمرین و مواردی که در کلاس عنوان شده،‌ ساختار شبکه و تابع خطا و پارامترهای دیگر را طوری طراحی و انتخاب نمایید که در نهایت تصاویر باکیفیتی توسط Generator تولید شود و فرایند آموزش پایدار باشد.
در نهایت فایل تکمیل شده به همراه نتایج را بعلاوه فایل پارامترهای شبکه Generator ای که آموزش داده‌اید به همراه دیگر بخش‌های تمرین ارسال نمایید.


\begin{flushleft}
	موفق باشید :)
\end{flushleft}



\پایان{نوشتار}
