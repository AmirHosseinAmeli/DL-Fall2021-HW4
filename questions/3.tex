\begin{enumerate}[label=(\alph*)]
	\item
	همانطور که می‌دانیم، دو مدل GAN و VAE از مهم ترین و شناخته شده ترین مدل‌های generative در یادگیری عمیق می‌باشند. یکی از مهم‌ترین کاربردهای آن‌ها، تولید تصاویر و augmentation data می‌باشد. در این صورت، با فرض استفاده از مجموعه داده یکسان و فرآیند آموزش نسبتا کامل، آیا به طور کلی کیفیت تصاویر تولید شده توسط یکی از این مدل‌ها بر دیگری برتری دارد؟ لطفا پاسخ خود را با دلیل و در صورت نیاز اثبات ریاضی تشریح نمایید.
	می‌توانید از 
	\href{https://arxiv.org/pdf/1706.01807.pdf}{این مقاله}
	استفاده نمایید.
	\item
	تابع ReLU یکی از پرکاربردترین توابع فعالسازی مورد استفاده در شبکه های یادگیری عمیق می‌باشد، اما در برخی کاربری‌های خاص همانند بخش Generative در روش GAN می‌تواند منجر به ایجاد مشکل در فرآیند آموزش شود، به عبارت دیگر شبکه مولد ما آموزش نخواهد دید. در این مورد استثنا، توصیه می‌شود به جای ReLU ، از ReLU Leaky به عنوان تابع فعالسازی استفاده گردد. لطفا علت بروز مشکل به هنگام استفاده از ReLU را به طور کامل توضیح داده و تشریح کنید که به چه علت استفاده از ReLU Leaky می‌تواند مشکل را حل کند.
	 
\end{enumerate}
