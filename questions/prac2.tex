
VAE Conditional یکی از ورژن‌های modified شده VAE بوده که بر خلاف VAE کلاسیک، متغیرهای مورد نیاز را به صورت conditioned نسبت به برخی متغیرهای تصادفی تخمین می‌زند. در این تمرین هدف مقایسه خروجی این دو مدل بر روی مجموعه داده MNIST می‌باشد. لطفا کد هر دو روش پیاده‌سازی شده و خروجی آن‌ها از بعد میزان وضوح تصاویر تولید شده مقایسه گردد.

برای طراحی شبکه های VAE و CVAE ، به طور کلی محدودیت چندانی وجود ندارد اما پیشنهاد می شود برای قسمت Encoder ، سه لایه کانولوشن دو بعدی به ترتیب با 16، 32، و 32 لایه به همراه MaxPool دو بعدی 2 در 2 پس از هرکدام طراحی شود. برای قسمت Decoder نیز دو لایه خطی به ترتیب با 32 و 64 لایه طراحی گردد. برای تایع هزینه لطفا از Entropy Cross Binary به همراه Divergence KL استفاده شود. برای Optimizer نیز از Adam استفاده گردد. مابقی پارامترها همانند mean می‌تواند به صورت customize شده انتخاب گردد و بسته به خروجی بهتر تغییر کند. برای راهنمایی بیشتر می‌توانید از کد موجود در
\href{https://github.com/chendaichao/VAE-pytorch}{این لینک}
استفاده کنید. لطفا کد را کپی نکرده و صرفا برای کمک و الهام‌گیری کد زنی خود از آن استفاده شود. توجه شود حتی ساختار پیشنهادی شبکه بسته به صلاح دید شخصی شما قابل تغییر بوده فقط توجه شود که نوشتن گزارش بخش عملی الزامی بوده و دارای نمره می‌باشد لذا حتما تمامی مراحل اعم از ساختار شبکه‌ها و پارامترها باید به طور کامل در گزارش توضیح داده شوند. برای پیاده‌سازی نیز تنها مجاز به استفاده از کتابخانه pytorch می‌باشید.
\\
\\


توضیحات کلی:


به وضوح مشورت و همفکری در حل سوالات هیچگونه ای مشکلی ایجاد نخواهد کرد اما جواب سوالات به هیچ عنوان نباید یکسان باشد. هر فرد باید جداگانه و Unique پاسخ‌های خود را بنویسد و در صورت شباهت بسیار زیاد نمره بین افرادی که پاسخ های بسیار مشابه دارند تقسیم خواهد گشت.
در صورتی که کد شما در بخش عملی به خطا برخورد و خطا قابل برطرف کردن نبود، نمره شما بر اساس کیفیت کد به میزان قابل قبولی لحاظ خواهد گشت. لذا لطفا پاسخ سوال‌های عملی را خالی نگذارید.
برای سوال عملی لطفا کد و گزارش هردو ارسال شوند. کمبود یکی از آن‌ها منجر به از دست رفتن نمره خواهد شد.

