در این سوال می خواهیم تاثیر تعداد داده های نمونه برداری شده را بر مقادیر بدست آمده توسط تخمینگرهای MLE و MAP بررسی کنیم.\\
یک توزیع به فرم گاوسی تک متغیره با پارامترهای 
$
x \sim N\left( {\mu  = 2,{\sigma ^2} = 5} \right)
$
را در نظر بگیرید که داده های موجود از این توزیع به صورت i.i.d نمونه برداری می‌شوند. همچنین فرض کنید توزیع پیشین $\mu $ یک توزیع گاوسی با پارامترهای 
$
N\left( {1,2} \right)
$
می باشد.\\
به ترتیب 10، 100 و 1000 نمونه از توزیع گاوسی اولیه نمونه بردارید. در هر حالت با کمک داده های نمونه برداری شده و توزیع پیشین بالا و روابط بدست آمده در مساله قبل، با استفاده از تخمین گرهای MLE و MAP مقدار $\mu $ را تخمین بزنید.\\
نتایج بدست آمده را تفسیر کنید. (به دلیل ذات تصادفی بودن نمونه های برداشته شده،‌ برای تفسیر دقیق‌تر توصیه می‌شود چندبار عمل بالا را تکرار نمایید.)