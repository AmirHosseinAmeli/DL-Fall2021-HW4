\begin{enumerate}[label=(\alph*)]
	\item

در یک VAE اگر داده ورودی از نوع باینری باشد (تصویری با پیکسل های 0 و 1)، می‌توان به جای توزیع گاوسی چندمتغیره روی خروجی کدگشا، از توزیع برنولی چندمتغیره استفاده کرد. توزیع خروجی کدگشا را به شکل برنولی چندمتغیره در نظر بگیرید و تابع فعال سازی آخرین لایه کدگشا را sigmoid در نظر بگیرید. اثبات کنید که در تابع هزینه این شبکه یک جمله ای به شکل Entropy Cross Binary ظاهر می شود.

	\item
	تکنیک Reparameterization روی بسیاری از توزیع‌های پیوسته قابل اعمال است. تحقیق کنید که چگونه می‌توان از این تکنیک برای یک توزیع categorical استفاده کرد؟
	
	\item
	یکی از نسخه‌های تغییریافته VAE مقاله مربوط به beta-VAE می باشد.  در این روش یک ضریب beta ای پشت یکی از توابع هزینه قرار میگیرد. درباره این روش تحقیق کنید و بیان کنید:
	
	\begin{enumerate}[label=\arabic*)]
		\item
		با انجام چه روندی از محاسبات،‌ این ضریب در تابع هدف این روش ظاهر می‌شود؟
		\item
		هذف از افزودن ضریب beta چیست و اضافه کردن آن چه تاثیری روی ویژگی‌های فضای نهان یادگرفته‌شده توسط مدل دارد؟
	\end{enumerate}
	
\end{enumerate}