\begin{enumerate}[label=(\alph*)]
	\item
	یکی از راه‌های معمول برای تخمین گرادیان یک امید ریاضی، استفاده از رابطه‌ی زیر است:
	
	$$
	{\nabla _\theta }{\mathbb{E}_{z \sim {q_\theta }\left( z \right)}}\left[ {f\left( z \right)} \right] = \frac{1}{N}\sum\limits_{i = 1}^N {f\left( {{z^i}} \right)} .{\nabla _{_\theta }}\ln {q_\theta }\left( {{z^i}} \right)
	$$
	
	که در آن هر $z^i$ نمونه‌ی مستقلی از توزیع ${{q_\theta }\left( z \right)}$ می باشد. درستی این رابطه را نشان دهید و بیان کنید که چطور می توانیم از آن در VAE استفاده کنیم.
	با مراجعه به
	\href{http://icml.cc/2012/papers/687.pdf}{این مقاله}
	مشکلی که در استفاده از این روش وجود دارد را بیان کنید.
	
	\item
	به صورت شهودی بیان کنید که روش Reparameterization چگونه می‌تواند این مشکل را حل کند؟
	
	 \item
	 در بسیاری از موارد تابع خطای رمزگشای VAE را خطای MSE در نظر میگیریم. این در حالی است که هدف ما بیشینه کردن تابع
	 $
	 {\mathbb{E}_{z \sim {q_\theta }\left( {z|x} \right)}}\ln {p_\theta }\left( {x|z} \right)
	 $
	 می باشد. در چه صورتی و با چه فرض هایی این دو کار معادل یکدیگر هستند؟
	 
\end{enumerate}